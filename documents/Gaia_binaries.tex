\documentclass[usenatbib]{mnras}
%\documentclass[12pt, preprint]{aastex}
\usepackage{bm}
\usepackage{amsmath, amssymb}
\usepackage{comment}
\usepackage{graphicx}

\newcommand{\setof}[1]{\left\{{#1}\right\}}
\newcommand{\given}{\,|\,}
\newcommand{\dd}{\mathrm{d}}
\newcommand{\catalog}{\bm{Q}}
\newcommand{\pars}{\bm{\theta}}
\newcommand{\degree}{\ifmmode {^\circ}\else$^\circ$\ \fi}
\newcommand{\amin}{\ifmmode {^{\prime}\ }\else$^{\prime}$\fi}
\newcommand{\asec}{\ifmmode {^{\prime\prime}}\else$^{\prime\prime}$\fi}
\newcommand{\bs}[1]{\boldsymbol{#1}}
\newcommand{\Msun}{\ifmmode {M_{\odot}}\else${M_{\odot}}$\fi}




\title[Gaia wide binaries]{Placeholder Title: Wide Binaries in the Gaia-Tycho 2 Catalog (or how I learned to love big data in astronomy)}
\author[J. J. Andrews et al.]{Jeff J. Andrews$^{1}$\thanks{Contact e-mail: \href{mailto:andrews@physics.uoc.gr}{andrews@physics.uoc.gr}}, Marcel Ag\"{u}eros$^2$, Julio Chanam\'{e}$^3$ \\
$^1$ IESL/Foundation for Research and Technology - Hellas \\
$^2$ Columbia University \\
$^3$ PUC} 

\begin{document}
\label{firstpage}
\pagerange{\pageref{firstpage}--\pageref{lastpage}}
\maketitle



\begin{abstract}
Abstract goes here. 
\end{abstract}

\section{Introduction}

There have been many efforts to search for wide binaries in the past. In the past, these have fallen under two categories depending on whether independent stars are matched using only position \citep{bahcall81, gould95, dhital15} or if proper motion is also included \citep{luyten79, wasserman91, dhital10}. The astrometric precision of Gaia will precisely measure astrometric parallax for some 10$^9$ stars, providing an unprecedented opportunity to identify wide binaries using five dimensions of phase space. Furthermore, by the final Gaia data release, its narrow-band spectrograph will obtain radial velocities for some subset of these stars allowing for a sample to be identified with fully six dimensions of phase space.


Notes:
\begin{itemize}
\item NLTT catalog originally from \citet{luyten79}. Revised by \citet{gould03} and \citet{salim03}. The rNLTT catalog contains some 36,000 stars. 
\item \citet{chaname04} identify 999 binaries within the rNLTT catalog, using position, proper motion, and reduced proper motion diagram.
\item Tycho-2 catalog contains some 2.5 million stars with position, proper motion and $B$ and $V$ magnitudes \citep{hog00b}.
\item \citet{fabricius02} put together the Tycho Double Star Catalog (TDSC) with 66,219 objects in 32,632 systems.
\item It takes roughly 10 measurements of a binary to determine if the motion is Keplerian or rectilinear \citep{fabricius02}.\\
\item TDSC Limiting separation is roughly 0.8\asec. Limiting magnitude is $V_T=11.5$ mag for primary \citep{fabricius02}.
\end{itemize}

The TDSC is composed of three different parts \citep{fabricius02}:
\begin{itemize}
\item Tycho double star solutions: Only calculated for pairs with separations less than 2.5\asec. Details can be found in \citet{hog00a}.\\
\item Tycho-2 stars in the Washington Double Star Catalog (WDS). Essentially a cross-correlation between stars in the Tycho-2 catalog and the WDS \citep{mason00}. Because the WDS catalog is so noisy and heterogeneous, lots of problems arose in this cross-correlation. Clearly there are spurious pairs in the WDS. \\
\item Tycho-2 pairs with separations less than 10\asec. Only 359 of these existed
\end{itemize}


The Gaia-Tycho-2 combined dataset contains some 2.5 million stars. The dataset is nearly complete to $V\approx$11 magnitude, but includes stars out to and exceeding $V\approx$12 mag. These stars have a typical astrometric uncertainties of order 300 $\mu$as in position, 700 $\mu$as in parallax, and 1.5 mas yr$^{-1}$ in proper motion. 





In this work, we describe a statistical method for identifying wide binaries using position, proper motion, and astrometric parallax. Although our discussion is focused on identifying pairs within the joint Gaia-Tycho-2 catalog, our method is derived to be flexibly adapted to future Gaia data releases, as well as future astrometric catalogs.

Previously, \citet{shaya11} outlined a method similar to the one here to identify wide binaries in the {\it Hipparcos} catalog. Our method here differs from and improves upon theirs in a number of important ways, which makes our method more suitable to {\it Gaia} catalogs. {\bf Outline their method in a sentence or two.}


\section{Statistical Method}


\begin{figure}
\begin{center}
\includegraphics[width=0.95\columnwidth]{../figures/tycho-2_pos_mu.pdf}
\caption{ Position and proper motion distribution of stars in the Tycho-2 catalog. The structure in the position distribution is strongly determined by the Galactic plane. Proper motions are typically smaller than 50 mas yr$^-1$, however a small portion of the sample have proper motions extending to several 10$^2$ mas yr$^{-1}$. {\bf Remake plots: Position plot is put on a better sky projection. Proper motion plot includes 1D histograms on top and right.}}
\label{fig:tycho-2_pos_mu}
\end{center}
\end{figure}


Figure \ref{fig:tycho-2_pos_mu} shows the position distribution (top panel) and proper motion distribution (bottom panel) of stars in the Tycho-2 catalog. The density of stars clearly varies depending on position and proper motion. It should be therefore be obvious that the probability of any two stars forming by a random alignment of two unassociated stars depends strongly on its position in phase space. At the same time, binary evolution theory provides loose expectations for the distribution of binaries we should expect. The goal in this section is to quantify the probabilities that any particular pair of stars is produced by both a random alignment and a true binary. Combining these probabilities gives us the Bayesian posterior probability that any pair of stars is a true binary. 



We consider any particular pair of stars occupying similar positions in five dimensional phase space ($\alpha$, $\delta$, $\mu_{\alpha}$, $\mu_{\delta}$, and $\pi$) to be formed from one of two classes, either a random alignment ($C_1$) or a genuine binary ($C_2$). This probability will, in general, depend on all five parameters (e.g., random alignments are more likely in dense stellar regions with low proper motions). To account for variations in this probability, we first assume that we are only interested in stellar pairs with close enough positions in phase space such that there is no difference in the average stellar density in the vicinity each star. This allows us to separate the two sets of astrometric parameters (one for each star) into $\vec{x}_j$, the pair's position in four-dimensional phase space, and $\vec{x}_i$, the difference in astrometric parameters between the two stars:
\begin{eqnarray}
\vec{x}_i &=& \{\theta, \Delta \mu', \pi'_1, \pi'_2 \} \\
\vec{x}_j &=& \{ \alpha, \delta, \mu_{\alpha}, \mu_{\delta} \},
\end{eqnarray}
where we use primes to indicate observed quantities that have some non-negligible uncertainty associated with them; the astrometric parallax is such that angular separation uncertainties can be ignored while proper motion difference and parallax uncertainties cannot. 

For two arbitrary stars, the components of $\vec{x}_j$ are simply the average of the measurements from each binary, weighted by their uncertainties. The components of $\vec{x}_i$ will be determined by the difference between the two measured values; each component will have an uncertainty associated with it. For closely separated stars the angular separation can be determined precisely from the two stellar coordinates:
\begin{equation}
\theta \approx \sqrt{(\alpha_A - \alpha_B)^2 \cos \delta_A \cos \delta_B
			 + (\delta_A - \delta_B)^2}.
\end{equation}
The proper motion difference can be similarly calculated:
\begin{equation}
\Delta \mu \approx \sqrt{(\mu_{\alpha, A} - \mu_{\alpha, B})^2 
			\cos \delta_A \cos \delta_B 
			+ (\mu_{\delta, A} - \mu_{\delta, B})^2}.
\end{equation}
Uncertainties in $\Delta \mu$ are separately calculated from standard error analysis:
\begin{equation}
\sigma_{\Delta \mu}^2 = \left( \sigma_{\mu_{\alpha, A}}^2 + \sigma_{\mu_{\alpha, B}}^2 \right) \cos \delta_A \cos \delta_B + \sigma_{\mu_{\delta, A}}^2 + \sigma_{\mu_{\delta, B}}^2  
\end{equation}



Using these sets of parameters and the two classes, we can now use Bayes's theorem to construct the generalized probability that any pair of stars forms a true binary (class $C_2$):
\begin{equation}
P(C_2 \given \vec{x}_i, \vec{x}_j) = \frac{P(\vec{x}_i \given C_2, \vec{x}_j) P(C_2 \given \vec{x}_j)}{P(
\vec{x}_i \given \vec{x}_j)}. \label{eq:P_binary_1}
\end{equation}
The first term in the numerator is a likelihood while the second term in the numerator can be considered a prior probability of a binary with $\vec{x}_j$. The denominator in Equation \ref{eq:P_binary_1} is the evidence, which can be determined by summing the probability that the particular pair can be produced by both a binary ($C_2$) and a random alignment ($C_1$):
\begin{equation}
P(\vec{x}_i \given \vec{x}_j) = \sum_{k=1,2} P(\vec{x}_i \given C_k, \vec{x}_j) P(C_k \given \vec{x}_j).
\end{equation}


If we assume the binary fraction ($f_{\rm bin}$) is a constant, we can express the conditional prior probabilities for $C_1$ and $C_2$ in terms of $P(\vec{x}_j)$, the density of stars as a function of phase space position $\vec{x}_j$:
\begin{eqnarray}
P(C_2 \given \vec{x}_j) &=& P(\vec{x}_j) f_{\rm bin} \\
P(C_1 \given \vec{x}_j) &=& P(\vec{x}_j).
\end{eqnarray}

We can combine these with Equation \ref{eq:P_binary_1}, to get:
\begin{equation}
P(C_2 \given \vec{x}_i, \vec{x}_j) = \frac{P(\vec{x}_i \given C_2, \vec{x}_j) f_{\rm bin} }{P(\vec{x}_i \given C_2, \vec{x}_j) f_{\rm bin}  + P(\vec{x}_i \given C_1, \vec{x}_j) }. \label{eq:P_binary_2}
\end{equation}


%The density for any particular position in phase space $n(\vec{x}_j)$ can be determined empirically by smoothing the number of star counts observed in the Gaia - TYCHO-2 sample. This function needs to be determined and calibrated using the actual catalog data, but one can in principle train on the simulated data set.


\subsection{Random Alignment Likelihood: $P(\vec{x}_i \given C_1, \vec{x}_j)$}

The probability that a pair of stars with $\vec{x}_i$ and $\vec{x}_j$ could have been formed due to random alignments is $P(\vec{x}_i \given C_1, \vec{x}_j)$. We begin by marginalizing over the true individual parallaxes $\pi_1$ and $\pi_2$ and the true proper motion difference $\Delta \mu$ to account for observational uncertainties on these quantities:
\begin{equation}
P(\vec{x}_i \given C_1, \vec{x}_j) = \int \dd \pi_1\ \dd \pi_2\ \dd \Delta \mu\ P(\pi_1, \pi_2, \Delta \mu, \vec{x}_i \given C_1, \vec{x}_j). \label{eq:P_noise_marginalized}
\end{equation}
Now we can substitute for $\vec{x}_i$ and $\vec{x}_j$ and factor out $\pi'_1$, $\pi'_2$, and $\Delta \mu'$ since observed quantities are dependent only on their underlying values (and their associated uncertainties):
\begin{eqnarray}
P(\vec{x}_i \given C_1, \vec{x}_j) &=& \int  \dd \pi_1\ \dd \pi_2\ \dd \Delta \mu\ P(\pi'_1 \given \pi_1) P(\pi_1) \nonumber \\
	& &  \times P(\pi'_2 \given \pi_2) P(\pi_2) P(\Delta \mu' \given \Delta \mu) \nonumber \\
	& &  \times  P(\theta, \Delta \mu \given C_1, \alpha, \delta, \mu_{\alpha}, \mu_{\delta}). \label{eq:RA_substituted}
\end{eqnarray}
Our assumption that $\theta$ and $\Delta \mu$ are independent allow us to split the last term in the integrand. 
%\begin{eqnarray}
%P(\vec{x}_i \given C_1, \vec{x}_j) &=& \int  \dd \pi_1\ \dd \pi_2\ \dd \Delta \mu\ P(\pi'_1 \given \pi_1) P(\pi_1) \nonumber \\
%	& &  \times P(\pi'_2 \given \pi_2) P(\pi_2) P(\theta \given C_1, \alpha, \delta) \nonumber \\
%	& &  \times  P(\Delta \mu' \given \Delta \mu) P(\Delta \mu \given C_1, \mu_{\alpha}, \mu_{\delta}). \label{eq:RA_substituted}
%\end{eqnarray}
Furthermore, the integrals over $\pi_1$ and $\pi_2$ are independent and can be factored out. Equation \ref{eq:RA_substituted} therefore becomes:
\begin{eqnarray}
P(\vec{x}_i \given C_1, \vec{x}_j) &=& \int P(\theta \given C_1, \alpha, \delta) P(\Delta \mu' \given \Delta \mu) \nonumber \\
	& & \times\ P(\Delta \mu \given C_1, \mu_{\alpha}, \mu_{\delta})\ \dd \Delta \mu \nonumber \\
	& &  \times \int  P(\pi'_1 \given \pi_1) P(\pi_1)\ \dd \pi_1 \nonumber \\
	& &  \times \int  P(\pi'_2 \given \pi_2) P(\pi_2)\ \dd \pi_2. \label{eq:RA_factored}
\end{eqnarray}

To determine the $\theta$ term in Equation \ref{eq:RA_factored}, we recognize that as $\theta$ increases, the probability of random alignments increases linearly. This scaling is because, for a uniformly dense population on the sky, consecutively larger annuli surrounding a star have areas that increase linearly, and the probability of random alignments should scale with the area (at least locally, so the density of stars per phase space is constant). The scaling coefficient depends on the local stellar density surrounding a point. We therefore determine this scaling empirically. 

To do this, we calculate the kernel density estimate (KDE), using a Gaussian kernel, for the distribution of stars in the Gaia-Tycho-2 catalog. The KDE acts as a smoothing function over the distribution in Figure \ref{fig:tycho-2_pos_mu} and produces a normalized probability density function over position space. {\bf Does this need to be calculated on $\alpha \cos \delta$ and $\delta$, instead of $\alpha$ and $\delta$?}

\begin{figure}
\begin{center}
\includegraphics[width=0.95\columnwidth]{../figures/tycho-2_local_density.pdf}
\caption{ For one particular star, the solid line in the top panel shows the normalized histogram of the distances to other stars in the Tycho-2 catalog. For a constant density (locally, at least) the distribution should rise linearly with angular separation. The dashed line shows the expectation from our KDE estimate using Equation \ref{eq:P_theta}. For the same star, the solid line in the bottom panel shows the equivalent normalized histogram of proper motion differences. For proper motion distributions beyond $\approx$10 mas yr$^{-1}$, the distribution substantially rises, deviating from linear. The dashed line again indicates our approximation using the KDE density estimate using Equation \ref{eq:P_mu}. Since typical proper motion uncertainties are $\lesssim$ 5 mas yr$^{-1}$, our linear approximation is reasonable. Distributions for other stars are similar to the ones shown here.}
\label{fig:tycho-2_density}
\end{center}
\end{figure}

The top panel of Figure \ref{fig:tycho-2_density} demonstrates the typical accuracy of our empirically-derived estimate for the local space density. This figure shows the distribution for one particular star representative of typical accuracies for our estimation methods. The solid line indicates the normalized histogram of stars at a distance $\theta$, while the dashed line approximates this distribution using our KDE estimate. The dashed line is determined independently for each primary object by calling the KDE function to determine the stellar density, as a function of $\alpha$ and $\delta$:
\begin{equation}
P(\theta \given C_1, \alpha, \delta) = 2 \pi \rho(\alpha, \delta)\ \theta. \label{eq:P_theta}
\end{equation}


The $\Delta \mu$ term in Equation \ref{eq:RA_factored} provides the corresponding effect for proper motion. This term can also be calculated empirically using a KDE, again with a Gaussian kernel. The bottom panel of Figure \ref{fig:tycho-2_density} demonstrates the accuracy of our estimation method for the number of stars with similar proper motions to the same star as in the top panel. The solid line shows the actual normalized histogram of proper motion difference $\Delta \mu$, while the dashed line shows the KDE estimate: 
\begin{equation}
P(\Delta \mu \given C_1, \alpha, \delta) = 2 \pi \rho_{\mu}(\mu_{\alpha}, \mu_{\delta})\ \Delta \mu, \label{eq:P_mu}
\end{equation}
where $\rho_{\mu}(\mu_{\alpha}, \mu_{\delta})$ is the proper motion-dependent density determined by the KDE.

Due to smaller scale variations in the proper motion, it is evident from Figure \ref{fig:tycho-2_density} that the KDE estimation method deviates from the true distribution for proper motion differences $>10$ mas yr$^{-1}$. Since typical binaries have $\Delta \mu \lesssim$ 5 mas yr$^{-1}$, the KDE method is accurate enough for our purposes here. 




Probability terms involving the observed quantities in Equation \ref{eq:RA_factored} can be substituted as normal distributions, based on the Gaussian uncertainties in these terms. Non-Gaussian uncertainties can be taken into account here as well:
\begin{eqnarray}
P(\Delta \mu' \given \Delta \mu) &=& \mathcal{N}(\Delta \mu \given \Delta \mu', \sigma^2_{\Delta \mu}) \label{eq:delta_mu_uncertainty} \\
P(\pi'_1 \given \pi_1) &=& \mathcal{N}(\pi_1 \given \pi'_1, \sigma^2_{\pi_1}) \label{eq:pi_1_uncertainty} \\
P(\pi'_2 \given \pi_2) &=& \mathcal{N}(\pi_2 \given \pi'_2, \sigma^2_{\pi_2}). \label{eq:pi_2_uncertainty} \\
\end{eqnarray}


{\bf Paragraph and plot on parallaxes. Do we need to consider the Lutz-Kelker bias in $P(\pi)$ here?}


The random alignment likelihood, $P(\vec{x}_i \given C_1, \vec{x}_j)$ can now be determined by substituting  Equations \ref{eq:P_theta}, \ref{eq:P_mu}, \ref{eq:delta_mu_uncertainty}, \ref{eq:pi_1_uncertainty}, and \ref{eq:pi_2_uncertainty} into the marginalized and factored Equation \ref{eq:RA_factored}. We calculate these integrals using Monte Carlo sampling; our convergence tests indicate that 10$^3$ samples provide a sufficient accuracy for our purposes here.




\begin{comment}
The last two integrals in Equation \ref{eq:RA_factored} account for uncertainties in each of the astrometric parallax measurements. We assume that Lutz-Kelker bias \citep{lutz73} needs to be taken into account, therefore:
\begin{equation}
P(\pi) \propto \pi^4.
\end{equation}
Observational uncertainties on the parallax are assumed to be Gaussian:
\begin{equation}
P(\pi' \given \pi) = \mathcal{N}(\pi'; \pi, \sigma_{\pi}). 
\end{equation}
\end {comment}


Equation \ref{eq:RA_factored}, with our empirical estimate based on the local stellar density for $P(\theta \given C_1, \alpha, \delta)$ and the KDE estimate for $P(\Delta \mu \given C_1, \alpha, \delta, \mu_{\alpha}, \mu_{\delta})$ then defined the likelihood for any particular pair of stars to be produced by random alignment.



\begin{comment}
With a detailed knowledge of the three-dimensional stellar density, proper motion distribution as a function of spatial position, and completeness function of the Gaia-TYCHO-2 sample, one could determine $P(\theta, \Delta \mu, \Delta D' \given C_1, \alpha, \delta, \mu_{\alpha}, \mu_{\delta}, D )$ for any arbitrary pair of stars. However, this is a function with complex dependencies that we are not prepared to address here. Instead, we can empirically approximate this probability for every pair of stars in our data set. First, we express $\mu$ as $V_{\rm pec}$ (to remove the correlation between proper motion and distance), and then separate the terms in the probability:
\begin{eqnarray}
P(\theta, \Delta \mu, \Delta D' \given C_1, \alpha, \delta, \mu_{\alpha}, \mu_{\delta}, D ) &=& P(\theta, \Delta V_{\rm pec}, \Delta D' \given C_1, \alpha, \delta, \mu_{\alpha}, \mu_{\delta}, D ) \left| \frac{\dd \Delta V_{\rm pec}}{\dd \Delta \mu} \right| \label{eq:P_obs_noise} \\
&=& P(\theta \given \alpha, \delta) 
  P(\Delta V_{\rm pec} \given \alpha, \delta, \mu_{\alpha}, \mu_{\delta}) \nonumber \\
  & & \qquad \times  P(\Delta D' \given \alpha, \delta, D) \left| \frac{\dd \Delta V_{\rm pec}}{\dd \Delta \mu} \right|. \label{eq:P_obs_noise_split} 
\end{eqnarray}

These terms can be separated so long as no correlation exists between them. There is likely to be some dependence between these terms; for instance at larger distances, farther from the Galactic plane, stars are more likely to be halo members with larger peculiar velocities. A more complex model could account for this, however here we consider such variations to have only a minor effect on the probability in Equation \ref{eq:P_obs_noise}. 

We can determine the three probabilities in Equation \ref{eq:P_obs_noise_split} empirically using the following procedure. First, we select an arbitrarily large radius around each star to search for matching pairs. We then numerically generate three normalized histograms, one for each observable: $\theta$, $\Delta V_{\rm pec}$, and $\Delta D'$. Interpolating along the histograms, for each observable, for each pair, provides the three probabilities in Equation \ref{eq:P_obs_noise_split}. What remains is to determine a large enough search radius, such that there are enough that even widely separated, genuine stellar pairs can be resolved and that there are enough stellar pairs such that the three histograms do not suffer from low number statistics. For separations too large, our assumption of independence in observables breaks down, and the split we made in Equation \ref{eq:P_obs_noise} is no longer justified. This search radius needs to be calibrated, but for now, we suggest that 1\degree should be sufficient.

Combining Equation \ref{eq:P_obs_noise_split} and \ref{eq:P_noise_marginalized} gives us the following:
\begin{eqnarray}
P(\vec{x}_i \given C_1, \vec{x}_j) &=& P(\theta \given \alpha, \delta)\ 
   P(\Delta V_{\rm pec} \given \alpha, \delta, \mu_{\alpha} \mu_{\delta})\ 
   \left| \frac{\dd \Delta V_{\rm pec}}{\dd \Delta \mu} \right| \nonumber \\
   & & \qquad \times  \int \dd \Delta D'\ P(\Delta D \given \Delta D')\ P(\Delta D' \given \alpha, \delta, D),
\end{eqnarray}
where we were able to move the conditional probabilities for $\theta$ and $\Delta V_{\rm pec}$ outside the integral over $\Delta D'$ because of their independence on distance.
\end{comment}




\subsection{Binary Likelihood: $P(\vec{x}_i \given C_2, \vec{x}_j)$}

We now determine the probability that a true binary could produce the observations, $P(\vec{x}_i \given C_2, \vec{x}_j)$. We begin by first accounting for quantities with observational uncertainties. We marginalize over the true parallax, the true proper motion difference, the tangential velocity difference, $\Delta V$, and projected physical separation, $s$. Both components of a true binary will have the same distance (the widest binaries have separations $<$1 pc, less than Gaia's nominal parallax distance uncertainty), so we need only marginalize over one parallax:
\begin{eqnarray}
P(\vec{x}_i \given C_2, \vec{x}_j) &=& \int \dd \Delta \mu\ \dd \pi\ \dd \Delta V\ \dd s\ \nonumber \\
& & \qquad \times P( \Delta \mu, \pi, \Delta V, s,\vec{x}_i \given C_2, \vec{x}_j ). \label{eq:P_binary_marginalized_1}
\end{eqnarray}
Observed quantities depend only on the underlying values. We assume that the population of binaries is independent of both position in the Galaxy and its peculiar velocity, so $s$ and $\Delta V$ depend only on $C_2$. Upon substitution for $\vec{x}_i$ and $\vec{x}_j$, Equation \ref{eq:P_binary_marginalized_1} becomes:  
\begin{eqnarray}
P(\vec{x}_i \given C_2, \vec{x}_j) &=& \int \dd \Delta \mu\ \dd \pi\ \dd \Delta V\ \dd s\ P(\Delta \mu' \given \Delta \mu) \nonumber \\
& & \qquad \times P(\pi'_1 \given \pi)\ P(\pi'_2 \given \pi)\ P(s, \Delta V \given C_2) \nonumber \\
& & \qquad \times P(\Delta \mu \given \Delta V, \pi)\ P(\theta \given s, \pi)\ P(\pi), \label{eq:P_binary_marginalized_2}
\end{eqnarray}
where we have 

The fifth and sixth terms in the integrand of Equation \ref{eq:P_binary_marginalized_2} are delta functions that account for the proper motion difference and projected separation:
\begin{eqnarray}
P(\Delta \mu \given \Delta V, \pi) &=& \delta \left[ F(\Delta V) \right] \\ 
P(\theta \given s, \pi) &=& \delta \left[ G(s) \right], 
\end{eqnarray}
where $F(\Delta V) = \Delta \mu - \Delta V \pi$ and $G(s) = \theta - s \pi$. 

These delta functions reduce the integrals over $\Delta V$ and $s$ in Equation \ref{eq:P_binary_marginalized_2}:
\begin{eqnarray}
P(\vec{x}_i \given C_2, \vec{x}_j) &=& \int \dd \Delta \mu\ \dd \pi\ P(\Delta \mu' \given \Delta \mu)\ P(\pi'_1 \given \pi) \nonumber \\
& & \qquad \times P(\pi'_2 \given \pi)\ P(s, \Delta V \given C_2)\ P(\pi) \nonumber \\
& & \qquad \times  \left| \frac{\dd F(\Delta V^*)}{\dd \Delta V} \right|^{-1} \left| \frac{\dd G(s^*)}{\dd s} \right|^{-1}, \label{eq:P_binary_marginalized}
\end{eqnarray}
where $\Delta V^*$ and $s^*$ are the roots to $F(\Delta V)$ and $G(s)$, respectively. These last two terms in the integrand both reduce to $\pi^{-1}$.




The first three terms of the integrand are the observed distributions which we assume are Gaussian:
\begin{eqnarray}
P(\pi'_1 \given \pi) &=& \mathcal{N}(\pi'_1 \given \pi, \sigma^2_{\pi}) \\
P(\pi'_2 \given \pi) &=& \mathcal{N}(\pi'_2 \given \pi, \sigma^2_{\pi}) \\
P(\Delta \mu' \given \Delta \mu) &=& \mathcal{N}( \Delta \mu' \given \Delta \mu, \sigma^2_{\Delta \mu} ).
\end{eqnarray}



The fourth term in Equation \ref{eq:P_binary_marginalized}, $P(s, \Delta V_{\rm pec} \given C_2)$, expresses the likelihood that a random binary would produce the observed projected separation and velocity difference. In general, this function depends on assumptions made about populations of binary stars; we assume a binary is completely determined by four parameters: the two stellar masses, $M_1$ and $M_2$, the orbital separation, $a$, and the eccentricity, $e$. We do not include any binary evolution interactions between the two stars. 

We draw random binary separations, $a$, from a distribution flat in log space and eccentricities, $e$, from a thermal distribution:
\begin{eqnarray}
P(a) \propto a^{-1} &;& a \in [10^4 R_{\odot}:10^6 R_{\odot}] \\
P(e) \propto e\ \ \ &;& e \in [0:1).
\end{eqnarray}
We further adopt random binary orientation angles to determine the distribution of the projected difference in orbital motions. We adopt prior distributions for these parameters. 

We determine the density of true binaries in $s-\Delta V$ space by randomly generating 10$^6$ binaries from the above distributions, with random orientation angles. Figure \ref{fig:P_binary} demonstrates the distribution of our randomly generated binaries in this space. The bottom and left axes indicate $s$ and $\Delta V$, respectively, while the top and right axes indicate $\theta$ and $\Delta \mu$, respectively, if the entire population were at a distance of 100 pc. Gaia, with its ability to discern between double stars with subarcsecond separations and its proper motion precision of $\sim$ 1 mas yr$^{-1}$ for the joint Gaia-Tycho-2 catalog, can indeed detect differences in the orbital motion of the components of these widely separated binaries.


From the population of binaries shown in Figure \ref{fig:P_binary}, we use a KDE with a Gaussian kernel to create a normalized probability density function in this space. Evaluating the KDE at a particular $s$ and $\Delta V$ provides $P(s, \Delta V_{\rm pec} \given C_2)$.


\begin{figure}
\begin{center}
\includegraphics[width=0.95\columnwidth]{../figures/theta_mu.pdf}
\caption{ The distribution of 10$^6$ simulated binaries in peculiar velocity vs.\ projected physical separation space. Typical spatial velocities are $\sim$km s$^{-1}$, corresponding to $\sim$mas yr$^{-1}$ at 100 pc. {\it Gaia} has a proper motion precision that can measure the difference between the two stars in a wide binary.  }
\label{fig:P_binary}
\end{center}
\end{figure}

The final term in Equation \ref{eq:P_binary_marginalized} is the Lenz-Kelker bias on the parallax measure. Equation \ref{eq:P_binary_marginalized} can then become:
\begin{eqnarray}
P(\vec{x}_i \given C_2, \vec{x}_j) &=& \int \dd \Delta \mu\ \dd \pi\ 
\mathcal{N}( \Delta \mu' \given \Delta \mu, \sigma^2_{\Delta \mu} ) \nonumber \\
& & \qquad \times\ \mathcal{N}(\pi'_1 \given \pi, \sigma^2_{\pi})\ 
	\mathcal{N}(\pi'_2 \given \pi, \sigma^2_{\pi}) \nonumber \\
& & \qquad \times\ P(s, \Delta V \given C_2)\ P(\pi)\ \pi^{-2}.
\label{eq:P_binary}
\end{eqnarray}
We solve these integrals simultaneously using Monte Carlo random sampling. Convergence tests indicate that 10$^3$ random samples provides sufficient accuracy for our purposes here.






\section{Application to the Revised NLTT Catalog} \label{sec:rNLTT}


To test our method, we apply this method to the revised NLTT (rNLTT) catalog \citep{gould03, salim03}, containing 36,085 entries with a typical proper motion uncertainty of several mas yr$^{-1}$. A comparison of the pairs we identify to the 999 wide binaries previously identified by \citet[][hereafter CG04]{chaname04} in the rNLTT catalog provides a test of our method. The comparison is imperfect, since the rNLTT sample lacks the parallax measurements within the Gaia-Tycho-2 catalog. To apply our method to this sample, we assume all stars in the sample are at a distance of 100 pc, yet as we will see below, despite such a clearly incorrect assumption, our method remains effective.


\begin{figure}
\begin{center}
\includegraphics[width=0.95\columnwidth]{../figures/rNLTT_pos_mu.pdf}
\caption{The top panel shows the distribution of rNLTT stars in right ascension and declination, while the bottom panel shows the distribution in proper motion. The donut shape in the proper motion distribution is due to the proper motion selection criteria for the NLTT catalog: by definition, stars in the NLTT catalog have a nominal proper motion of at least 200 mas yr$^{-1}$. This sample is clearly the tail of the proper motion distribution of all nearby, bright stars.}
\label{fig:rNLTT_pos_mu}
\end{center}
\end{figure}



There are some important differences between the two search methods that deserve mention. First, CG04 identified binaries by first making simple proper motion cuts, using two separate $\Delta \mu$ limits for disk and halo binaries. Proper motion uncertainties are ignored, whereas the method defined here both includes uncertainties in proper motion as well as a quantification of binary probability as a function of $\theta$, $\Delta \mu$, and position in stellar position$-$proper motion phase space. However, to remove potential contaminants, CG04 include a constraint using an adapted version of the reduced proper motion to discern between stellar disk and halo populations. 
%\begin{equation}
%\eta = V + 5 \log \mu - 3.1(V-J) - 1.47 | \sin b | - 2.73. \label{eq:eta_rpm}
%\end{equation}
Clearly, true pairs cannot contain stars from different populations, a powerful constraint that our method ignores (Gaia-Tycho-2 parallaxes will make this constraint obsolete). Given the differences in these two methods, pairs disjoint between the two samples are to be expected.



Since we set the distance to every star in the catalog to 100 pc, our method greatly simplifies. $\vec{x}_i$ is redefined as $\vec{x}_i = \{ \theta, \Delta \mu' \}$. Equation \ref{eq:RA_substituted} for $P(\vec{x}_i \given C_1, \vec{x}_j)$ becomes:
\begin{eqnarray}
P(\vec{x}_i \given C_1, \vec{x}_j) &=& \int \dd \Delta \mu\ P(\Delta \mu' \given \Delta \mu)\nonumber \\
& & \qquad \times P(\theta \given C_1, \alpha, \delta) P(\Delta \mu \given C_1, \alpha, \delta, \mu_{\alpha}, \mu_{\delta}) \nonumber \\
\label{eq:P_random_NLTT}
\end{eqnarray}


Equation \ref{eq:P_binary} for $P(\vec{x}_i \given C_2, \vec{x}_j)$ becomes:
\begin{equation}
P(\vec{x}_i \given C_2, \vec{x}_j) = \int \dd \Delta \mu\ 
\mathcal{N}( \Delta \mu' \given \Delta \mu, \sigma^2_{\Delta \mu} )\
P(s, \Delta V \given C_2),
\label{eq:P_binary_rNLTT}
\end{equation}
where $s = \theta D$ and $V = \Delta \mu D$. 


{\bf Following paragraphs summarize different results between our method and Chaname \& Gould (2004). THESE PARAGRAPHS NEED TO BE UPDATED.}

After applying our method to the rNLTT catalog, we find similar results to CG04. Of their 999 identified pairs, we find 934 counterparts in the rNLTT catalog. Our method redetects 744 of this with a posterior probability greater than 50\%. 75 of the CG04 binaries were not detected because they have a $P(\vec{x}_i \given C_2, \vec{x}_j)$ value of zero. The other 115 non-detected CG04 pairs had nonzero binary probabilities, but had posterior probabilities less than 50\%. 



In addition to the 746 redetections, our method identifies an additional 242 pairs of stars with probabilities above 50\%. 44 of these are in the CG04 catalog, but ultimately ruled out as false positives for various reasons (22 because of ``uncertain classification", 15 because of ``inconsistent RPM positions", and seven because of ``large proper motion differences"). Four detected pairs were components of pairs in the CG04 catalog, but identified with different companions by our method; each of these four were found to be part of a triple system, in which our method matched a different pair of stars within the triple. The remaining 194 pairs seem to be true binaries not detected by CG04. 



Gray points in each panel of Figure \ref{fig:NLTT_probabilities} show the distribution of $P(\vec{x}_i \given C_1, \vec{x}_j)$ and $P(\vec{x}_i \given C_2, \vec{x}_j)$ for the pairs we identify. Comparison with the dashed lines corresponding to posterior probabilities of 50\% and 90\% indicate that the majority of these binaries are detected with very high significance. 

Of the sample identified by CG04, we redetect 707 pairs ($\approx 70$\%) with a posterior probability greater than 50\%. Colored points in the four panels of Figure \ref{fig:NLTT_probabilities} compare the pairs in CG04 with all 946 pairs that we identify, indicated as gray points. The top left panel shows the disk binaries, the top right panel shows the binaries containing a WD, the bottom left panel shows the halo binaries, and the bottom right binary shows those binaries detected but ultimately rejected by CG04 for reasons provided in the legend. Differences between the two samples are substantial.  

\begin{figure*}
\begin{center}
\includegraphics[width=0.95\textwidth]{../figures/rNLTT_P_binary_P_random.pdf}
\caption{ {\bf Unclear if this is the best figure to use to compare our pairs with those in Chaname \& Gould (2004). It is clear there will be differences, so we should focus on what about those differences we want to highlight.} Likelihoods of stellar pairs being produced by a true binary and a random alignment. Gray points indicate the 949 pairs with posterior probabilities of being a true binary greater than 50\%. Colored points in the four panels indicate disk binaries (top left), WD binaries (top right), halo binaries (bottom left), and binaries identified but rejected by CG04 (bottom right). }
\label{fig:NLTT_probabilities}
\end{center}
\end{figure*}







%\begin{figure}[h!]
%\begin{center}
%\includegraphics[width=0.95\columnwidth]{../figures/}
%\caption{ }
%\label{fig:test_corner}
%\end{center}
%\end{figure}

\section{Application to the Tycho-2 Catalog}

We apply our method to the Tycho-2 catalog, prior to its astrometric updates from {\it Gaia}. Figure \ref{fig:tycho-2_color_mag} demonstrates a distribution of a subset of the stars in the Tycho-2 catalog. We also indicate colors corresponding to different Main Sequence stars (black, dashed) and lines of constant distance (red, dashed) for Main Sequence stars. The bulk of the Main Sequence stars in the Tycho-2 catalog are at distances exceeding 100 pc. As with the rNLTT catalog, we lack astrometric parallax measurements, therefore we set all stars to a distance of 100 pc. Since the probability of any pair of stars being a true binary increases with closer distances, and Figure \ref{fig:tycho-2_color_mag} indicates that most stars are at a distance greater than 100 pc, our approximation overestimates the binary probability of pairs.


From the $2.5\times10^6$ stars in the catalog, we select those that have measured proper motions, leaving us with $\sim2\times10^6$ stars to search from. This set is nearly two orders of magnitude larger than the rNLTT catalog, and designing a computationally efficient algorithm is crucial to making the identification of wide binaries a tractable problem. Our method is as follows. For every star, we select each of the nearby stars within 1\degree. For typical stellar densities in the Tycho-2 sample, this corresponds to $\approx50$ stars. For each of these potential pairs, we perform a crude calculation of $P(\vec{x}_i \given C_2, \vec{x}_j)$: we use a Monte Carlo integration with only 10 random samples to calculate the integral in Equation \ref{eq:P_binary_rNLTT}. Pairs with a binary likelihood of zero are removed. For pairs with non-zero binary likelihoods resulting from our crude integration, we then calculate the binary likelihood (using 10$^4$ Monte Carlo samples for the integral of $\Delta \mu$) and the random alignment likelihood. We select pairs with posterior probabilities of being a binary at greater than 50\%.


Although seemingly crude, our initial integration using only 10 random samples is designed to quickly remove the vast majority of non-matching pairs that lie entirely in zero probability regions of $\theta-\Delta \mu$ parameter space. These integrations are the most computationally expensive aspect of the algorithm. Some genuine binaries may be missed by this process (false negatives); however in practice this method is extremely efficient at reducing the sample to focus on those binaries of interest. True binaries lost by only using 10 samples tend to lie at the edge of the true binary distribution, and end up as marginal classifications. {\bf Include more extensive testing on a subsample of Tycho-2 catalog.}

 
 
 
\begin{figure}[h!]
\begin{center}
\includegraphics[width=0.95\columnwidth]{../figures/tycho2_colors.pdf}
\caption{ The color-magnitude distribution of a random subset of the Tycho-2 catalog. Vertical dashed lines (black) indicate colors roughly corresponding to the stellar types labeled at the bottom of the figure for Main Sequence stars. Since reddening is ignored, these are only approximate. Slanted dashed lines (red) indicate corresponding distances for Main Sequence stars. The bulk of the Main Sequence stars are at distances exceeding 100 pc.}
\label{fig:tycho-2_color_mag}
\end{center}
\end{figure}





\section{Application to the Gaia-Tycho-2 Catalog}

{\bf This will be the meat of the paper.}

\subsection{Identifying Wide Binaries}
{\bf Apply our method to the Gaia-Tycho-2 catalog.}

\subsection{Determining Stellar Type/Classification}
{\bf Binaries with similar types (twin binaries). Binaries with subdwarfs or giants. WD wide binaries. Provide these distributions of stellar combinations, maybe as a table?}

\subsection{Specific Nearby Wide Binaries}
{\bf Redetect binaries in \citet{shaya11}? Find other nearby binaries that we should list specifically.}

\subsection{Identified Triple Systems} 
{\bf We already know there are some. Let's focus on just astrometric triples - there will be lots of spectroscopic binaries as components of our wide binaries, but this will require substantial cross matching, I think.}

\subsection{Angular and Physical Separation of Binaries}
{\bf Raw counts/distributions here will be interesting enough. Accounting for luminosity or method-biases will be very annoying and difficult, so leave this for a future work? Separation distributions as a function of stellar type would be interesting.}




\section{Discussion}

\subsection{Wide binary populations}
{\bf Identify kinematically different populations: Halo vs. disk?}

\subsection{Addition of {\it Gaia} data}
{\bf Compare sample found using full Gaia-Tycho-2 catalog (Section 5) with sample just from Tycho-2 position, proper motions (Section 4).  What does parallax (and better astrometric precision) add? Is parallax better at adding false negatives or removing false positives? This could be important for understanding observational biases in other astrometric wide binary samples.}

\subsection{The Widest binaries}
{\bf We are, in principle, sensitive to *extremely* wide binaries using this method. If there's anything near a parsec, or even a tenth of a parsec, we should discuss. These tell us about distributions of stars, giant molecular clouds, the Galactic tide, and dark matter distribution of the Milky Way. }

\subsection{Application to Future Gaia Data Releases}
{\bf 10$^9$ stars with full 5-D astrometry. What about subset with 6-D astrometry? Discuss computation expense challenges. Speculate on methods of increasing algorithm efficiency. Extrapolate this sample to estimate numbers of binaries expected from full sample. } 





\section{Conclusions}

{\bf Overview of our approach and results.}

{\bf New data will be coming soon (2017).}

{\bf Mention some ideas about science that can be done with large wide binary data sets. In particular, can we get a luminosity-unbiased sample, or at least account for observational biases (and biases introduced by detection methods).}


\section*{Acknowledgements}
Acknowledgements:
It is a pleasure to thank...
Funding...
Code...


\bibliographystyle{mnras}
\bibliography{references}

\label{lastpage}

\end{document}
