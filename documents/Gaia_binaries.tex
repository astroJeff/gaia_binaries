\documentclass[usenatbib]{mnras}
%\documentclass[12pt, preprint]{aastex}
\usepackage{bm}
\usepackage{amsmath, amssymb}
\usepackage{comment}
\usepackage{graphicx}

\newcommand{\setof}[1]{\left\{{#1}\right\}}
\newcommand{\given}{\,|\,}
\newcommand{\dd}{\mathrm{d}}
\newcommand{\catalog}{\bm{Q}}
\newcommand{\pars}{\bm{\theta}}
\newcommand{\degree}{\ifmmode {^\circ}\else$^\circ$\ \fi}
\newcommand{\amin}{\ifmmode {^{\prime}\ }\else$^{\prime}$\fi}
\newcommand{\asec}{\ifmmode {^{\prime\prime}}\else$^{\prime\prime}$\fi}
\newcommand{\bs}[1]{\boldsymbol{#1}}
\newcommand{\Msun}{\ifmmode {M_{\odot}}\else${M_{\odot}}$\fi}
\newcommand{\Rsun}{\ifmmode {R_{\odot}}\else${R_{\odot}}$\fi}



\title[Gaia wide binaries]{Placeholder Title: Wide Binaries in the Gaia-Tycho 2 Catalog (or how I learned to love big data in astronomy)}
\author[J. J. Andrews et al.]{Jeff J. Andrews$^{1}$\thanks{Contact e-mail: \href{mailto:andrews@physics.uoc.gr}{andrews@physics.uoc.gr}}, Marcel Ag\"{u}eros$^2$, Julio Chanam\'{e}$^3$ \\
$^1$ IESL/Foundation for Research and Technology - Hellas \\
$^2$ Columbia University \\
$^3$ PUC} 

\begin{document}
\label{firstpage}
\pagerange{\pageref{firstpage}--\pageref{lastpage}}
\maketitle



\begin{abstract}
We perform a search through the Tycho-2$-$Gaia astrometric catalog (TGAS) for wide stellar binaries by matching position, proper motion, and astrometric parallax. Our method to separate genuinely associated stellar pairs with randomly aligned, unassociated double stars uses a fully Bayesian formulation that accounts for uncertainties in both the proper motion and parallax. Furthermore, we design our method for scalability for expansion to future Gaia data sets which will be three orders of magnitude larger. After testing our method against previously identified sets of wide binaries, we apply it to the TGAS catalog and identify {\bf XXX} stellar binaries with a posterior probability greater than 90\%. Cross-matching the subset of these pairs also found in the RAVE radial velocity survey provides an estimate of the fidelity of our catalog; we find the vast majority of the candidate binaries in RAVE have consistent radial velocities indicating the effectiveness of our method. We identify several triple systems, as well as a number of nearby wide binaries. We confirm/reject the previous characterization of the visual stars {\bf XXX} as being in wide binaries.
\end{abstract}


\section{Introduction}

Roughly half of all Main Sequence stars are found in binary systems with orbital separations extending from tens of \Rsun\ to $\sim$ pc. Unresolved binaries at the smallest separations are typically found by identifying radial velocity variations indicative of orbital motion, while binaries separated at larger separations can be identified by observing the star's motion through its orbit over periods of years. At the largest separations two stars are typically associated using by watching the stars move through the stellar neighborhood in the same direction and rate. These so-called common proper motion pairs have typical separations between 10$^2$ and 10$^5$ AU, a large fraction of all binary stars.


Beyond $\sim10^3$ AU, binary system are unexpected to have any appreciable interaction throughout their lifetime {\bf cite Dylan's paper here}. At the same time, such binaries coeval since most formation scenarios indicate they must have been formed out of the dissolution of stellar clusters {\bf Citations from Julio here}. As coeval, independently-evolving stars, wide binaries are considered mini-clusters \citep{soderblom10}, allowing for a unique accessibility to answer questions unaddressable by other stellar systems.


Wide binaries have been used extensively to calibrate stellar aging methods such as lithium depletion {\bf citations from Soderblom}, subdwarfs with MS pairs to calibrate stellar aging {\bf Julio's papers}, WDs with MS pairs to calibrate the IFMR {\bf Zhong and Oswalt}, and double white dwarfs to calibrate the IFMR \citep{finley97, andrews15}. At the widest separations, binaries are affected by interactions with other stars, giant molecular clouds, and the Galactic tide, each of which acts to stochastically disrupt the pair \citep{weinberg87, jiang10}. Characterizing the sample at the widest separations therefore has the potential to constrain underlying physics of the Milky Way's structure. Constraints made by wide binaries are typically limited by their statistics; identifying large numbers of wide binaries is difficult, and often such samples suffer from significant and hard-to-characterize observational biases.


Previous efforts identifying wide binaries have focused on individual astrometric catalogs such as Luyten two tenths catalog \citep{luyten79}, {\bf list catalogs here} \citep{bahcall81, gould95, wasserman91}, the rNLTT catalog \citep{chaname04}, SDSS \citep{dhital10, dhital15, longhitano10, andrews12, baxter14, andrews15}, and Hipparcos \citep{shaya11}. These searches typically fall under two categories depending on whether independent stars are matched using only position or whether proper motion is included as well, corresponding to matching two or four dimensions, respectively, of the six-dimensional phase space. 


\citep{shaya11} perform the first large-scale with the additional inclusion of astrometric parallax from Hipparcos astrometry. The additional phase space dimension adds a powerful constraint to separate genuine binaries from randomly aligned stellar pairs, however because of its non-linear dependence on distance and non-trivial prior probabilities, parallax adds a significant degree of complexity to matching stellar pairs. Nevertheless, using a Bayesian formalism, these authors identify hundreds of pairs including some with projected separations of $\sim$pc with high probability. Clearly the inclusion of parallax offers the opportunity to identify an astrophysical interesting set of wide binaries.


The first data release from the Gaia satellite \citep{lindegren16} includes some 2.5 million stars in the TGAS catalog which matches objects detected by Gaia with objects in the Tycho-2 catalog \citep{hog00}. The catalog is nearly complete to $V=11$ mag, and includes stars out to and exceeding $V\approx$12 mag. Each of these stars has positions (with uncertainties of {\bf XXX} $\mu$asec), proper motions (with uncertainties $\approx$ 1.3 mas yr$^{-1}$), and parallaxes (with uncertainties $\approx 0.3$ mas). 


In this work, we describe a statistical method for identifying wide binaries using position, proper motion, and astrometric parallax. Although our discussion is focused on identifying pairs within the TGAS catalog, our method is designed to be scalable for future Gaia data releases.


%The astrometric precision of Gaia will precisely measure astrometric parallax for some 10$^9$ stars, providing an unprecedented opportunity to identify wide binaries using five dimensions of phase space. Furthermore, by the final Gaia data release, its narrow-band spectrograph will obtain radial velocities for some subset of these stars allowing for a sample to be identified with fully six dimensions of phase space.

\begin{comment}
Notes:
\begin{itemize}
\item NLTT catalog originally from \citet{luyten79}. Revised by \citet{gould03} and \citet{salim03}. The rNLTT catalog contains some 36,000 stars. 
\item \citet{chaname04} identify 999 binaries within the rNLTT catalog, using position, proper motion, and reduced proper motion diagram.
\item Tycho-2 catalog contains some 2.5 million stars with position, proper motion and $B$ and $V$ magnitudes \citep{hog00b}.
\item \citet{fabricius02} put together the Tycho Double Star Catalog (TDSC) with 66,219 objects in 32,632 systems.
\item It takes roughly 10 measurements of a binary to determine if the motion is Keplerian or rectilinear \citep{fabricius02}.\\
\item TDSC Limiting separation is roughly 0.8\asec. Limiting magnitude is $V_T=11.5$ mag for primary \citep{fabricius02}.
\end{itemize}

The TDSC is composed of three different parts \citep{fabricius02}:
\begin{itemize}
\item Tycho double star solutions: Only calculated for pairs with separations less than 2.5\asec. Details can be found in \citet{hog00a}.\\
\item Tycho-2 stars in the Washington Double Star Catalog (WDS). Essentially a cross-correlation between stars in the Tycho-2 catalog and the WDS \citep{mason00}. Because the WDS catalog is so noisy and heterogeneous, lots of problems arose in this cross-correlation. Clearly there are spurious pairs in the WDS. \\
\item Tycho-2 pairs with separations less than 10\asec. Only 359 of these existed
\end{itemize}

\end{comment}


%The Gaia-Tycho-2 combined dataset contains some 2.5 million stars. The dataset is nearly complete to $V\approx$11 magnitude, but includes stars out to and exceeding $V\approx$12 mag. These stars have a typical astrometric uncertainties of order 300 $\mu$as in position, 700 $\mu$as in parallax, and 1.5 mas yr$^{-1}$ in proper motion. 


%Previously, \citet{shaya11} outlined a method similar to the one here to identify wide binaries in the {\it Hipparcos} catalog. Our method here differs from and improves upon theirs in a number of important ways, which makes our method more suitable to {\it Gaia} catalogs. {\bf Outline their method in a sentence or two.}




\subsection{Motivation \& Challenges}
\label{sec:motivation}

Gaia will ultimately astrometrically measure some 10$^9$ binaries, an increase of four orders of magnitude over Hipparchos. At the same time, the astrometric precision will increase by an order of magnitude. The sheer scale and precision of the data set means that identifying wide binaries in Gaia presents a unique data science challenge.


Hipparcos provides the only comparable data set to the currently available Gaia data set. This contains some 10$^5$ stars. However the number of stellar pairs of a catalog of size $N$ is the arithmetic mean $N(N-1)/2$, and therefore there are $\sim5\times10^9$ individual pairs of stars in the Hipparcos data set. While the TGAS catalog is 20 times larger, the total number of stellar pairs is increased to $2\times10^{12}$, a factor of 500 increase over Hipparcos. The final Gaia catalog will contain some $5\times10^{17}$ stellar pairs; clearly the same algorithms developed to search previous astrometric catalogs, even those developed to include parallax for Hipparcos, need to be rethought. The sheer scale of the problem is formidable.


Furthermore, the astrometric precision of Gaia presents its own unique problems for identifying wide binaries. Traditional methods have worked toward identifying pairs of close stars with identical proper motions; however, Gaia's astrometric precision is fine enough that matching stars with identical proper motions will miss a large portion of true binaries. We demonstrate this effect by generating a set of 10$^4$ random binaries. We draw random binary separations, $a$, from a distribution flat in log space and eccentricities, $e$, from a thermal distribution:
\begin{eqnarray}
P(a) \propto a^{-1} &;& a \in [10^4 R_{\odot}:10^6 R_{\odot}] \\
P(e) \propto e\ \ \ &;& e \in [0:1).
\end{eqnarray}
We further adopt random binary orientation angles to determine the distribution of the projected separation, $s$, and projected difference in orbital velocities, $\Delta V$. 

Figure \ref{fig:gaia_binary_limits} compares the distribution of binaries in $\log s-\log \Delta V$ space to Gaia's astrometric limits. The current TGAS data release can identify pairs of stars with angular separations of $\approx1$\asec and proper motions of $\approx 1.3 $ mas yr$^{-1}$. We show these regions in the top panel for three different distances; Gaia can detect the differential orbital motion for binaries within the boxed regions. By the final Gaia data release, the astrometry will become substantially more precise; on-ground processing may allow identification of stellar pairs with separations as small as 0.1\asec \citep{harrison11}, and proper motion precision may reach precisions of $\approx0.25$ mas yr$^{-1}$. For most of the wide binaries within 100 pc, and a substantial fraction within 500 pc, this differential proper motion must be taken into account.


\begin{figure}
\begin{center}
\includegraphics[width=0.95\columnwidth]{../figures/Gaia_s_dV_limits.pdf}
\caption{ The KDE representation of the density of 10$^4$ randomly produced binaries in $\log s- \log(\Delta V)$ space. A population of binaries with separations nearing 1 pc exist, but these should have identical proper motions. At somewhat smaller projected separations, the tangential velocities may differ by as much as tens of km s$^{-1}$. The top panel shows the approximate current sensitivity of Gaia of 1.3 mas yr$^{-1}$ in proper motion and 1\asec in angular separation for three different distances. Binaries within the boxed regions can be detected, but the difference in their tangential velocity can be detected by Gaia. Differential velocities due to orbital motion can be ignored for binaries below the boxed regions. The bottom panel shows that by the final Gaia data release, differential proper motions should be detectable for a much larger parameter space. }
\label{fig:gaia_binary_limits}
\end{center}
\end{figure}

While demonstrating a clear challenge to wide binary identification methods, Figure \ref{fig:gaia_binary_limits} also shows one of the primary motivations for going through the exercise: {\it most binaries exist at large separations}. Such binaries have separations far too large to observe orbital motion, and the only currently robust method to associate wide pairs involves matching their position in phase space. Previous surveys were constrained to only those nearby pairs with large proper motions. With Gaia's proper motion precision and parallaxes, we now have the ideal data set to access this region of binary parameter space. 


Finally, Gaia presents an additional, unprecedented opportunity to access all six dimensions of phase space. Its final radial velocity catalog will achieve precision of $\sim 10$ km s$^{-1}$, precise enough to substantially aide in separating genuine stellar binaries from randomly aligned stellar pairs. To our knowledge, radial velocities have exclusively been used in the past to check the binarity of wide binary candidates; no algorithms have yet been included within common proper motion pair searches. For the first time, large numbers of binaries will be able to be identified and confirmed by their full three-dimensional positions and velocities. Orbital motion will not be needed to claim binarity with certainty.


\section{Statistical Method}


\begin{figure}
\begin{center}
\includegraphics[width=0.95\columnwidth]{../figures/tycho-2_pos_mu.pdf}
\caption{ Position and proper motion distribution of stars in the Tycho-2 catalog. The structure in the position distribution is strongly determined by the Galactic plane. Proper motions are typically smaller than 50 mas yr$^-1$, however a small portion of the sample have proper motions extending to several 10$^2$ mas yr$^{-1}$. {\bf Remake plots: Position plot is put on a better sky projection. Proper motion plot includes 1D histograms on top and right.}}
\label{fig:tycho-2_pos_mu}
\end{center}
\end{figure}


Figure \ref{fig:tycho-2_pos_mu} shows the position distribution (top panel) and proper motion distribution (bottom panel) of stars in the Tycho-2 catalog. The density of stars clearly varies depending on position and proper motion. It should be therefore be obvious that the probability of any two stars forming by a random alignment of two unassociated stars depends strongly on its position in phase space. At the same time, binary evolution theory provides loose expectations for the distribution of binaries we should expect. The goal in this section is to quantify the probabilities that any particular pair of stars is produced by both a random alignment and a true binary. Combining these probabilities gives us the Bayesian posterior probability that any pair of stars is a true binary. As the next three subsections are somewhat technical, readers may wish to skip to Section \ref{sec:demonstrations} for a demonstration of our method.



\subsection{Overview of the Method}


Figure \ref{fig:example_s_dV} looks at two separate binaries analyzed by our method; the top panel shows a pair identified by our method as a true binary, while the bottom panel shows a pair that our method determines is a random alignment. The middle panels compare the astrometric parallaxes (based on their uncertainties) of the two stars in each pair as red and blue Gaussian curves, while the right panels show the corresponding proper motion uncertainties. Concentric circles indicate 1$-\sigma$, 2$-\sigma$, and 3$-\sigma$ regions. Although the system in the bottom panel clearly has a more inconsistent set of parallaxes and proper motions, they are still consistent at the $\approx$2$-\sigma$ level. 


\begin{figure*}
\begin{center}
\includegraphics[width=0.95\textwidth]{../figures/Examples_s_dV.pdf}
\caption{We compare the position in $\log s-\log \Delta V$ space of two pairs of stars from the TGAS catalog (black contours) to the expectation of a population of binary stars generated by the method described in Section \ref{sec:motivation} (blue background). The top panel shows a stellar pair identified by our method as a true binary, while the bottom panel shows a stellar pair rejected by our method. The contours, representing 1$-\sigma$, 2$-\sigma$, and 3$-\sigma$ confidence levels are created by accounting for uncertainties in the parallax and proper motions of the stars in each pair. Contours from the pair in the top panel overlap with the region of parameter space expected from genuine binaries, while contours in the bottom panel are clearly disparate. }
\label{fig:example_s_dV}
\end{center}
\end{figure*}


Our method separates genuine binaries from randomly aligned pairs by transforming from the pair's angular separation and each star's parallaxes and proper motions to projected physical separation and the difference in the transverse velocity. The distribution in these two parameters is indicated by the contours in the left-most panels. Contours again indicate 1$-\sigma$, 2$-\sigma$, and 3$-\sigma$ confidence levels. The blue background in these panels is the expected distribution of binaries described in \ref{sec:motivation}. The convolution of these two distributions, along with the Jacobian determinant term and parallax terms provided in Equation \ref{eq:P_binary} provides the probability that these pairs are consistent with being a true binary. 


This is one part of the calculation in a Bayesian formalism; our method also needs to identify the probability that any particular pair is consistent with a random alignment of stars. 


Finally, we need to calculate the prior probabilities of any particular pair being a genuine binary and a random alignment of stars. Since the number of stellar binaries in a sample scales with the size of the sample, $N$, while the number of randomly aligned pairs scales with $N(N-1)/2$, any particular pair has a very strong prior of being a randomly alignment. The exact value of this prior, which depends on the local density of each pair's position and proper motion, needs to be determined for every pair individually.



\subsection{Constructing the Bayesian Formalism}


We consider any particular pair of stars occupying similar positions in five dimensional phase space ($\alpha$, $\delta$, $\mu_{\alpha}$, $\mu_{\delta}$, and $\pi$) to be formed from one of two classes, either a random alignment ($C_1$) or a genuine binary ($C_2$). This probability will, in general, depend on all five parameters (e.g., random alignments are more likely in dense stellar regions with low proper motions). To account for variations in this probability, we first assume that we are only interested in stellar pairs with close enough positions in phase space such that there is no difference in the average stellar density in the vicinity each star. This allows us to separate the two sets of astrometric parameters (one for each star) into $\vec{x}_j$, the pair's position in four-dimensional phase space, and $\vec{x}_i$, the difference in astrometric parameters between the two stars:
\begin{eqnarray}
\vec{x}_i &=& \{\theta, \Delta \mu', \pi'_1, \pi'_2 \} \\
\vec{x}_j &=& \{ \alpha, \delta, \mu_{\alpha}, \mu_{\delta} \},
\end{eqnarray}
where we use primes to indicate observed quantities that have some non-negligible uncertainty associated with them; the astrometric parallax is such that angular separation uncertainties can be ignored while proper motion difference and parallax uncertainties cannot. 

For two arbitrary stars, the components of $\vec{x}_j$ are simply the average of the measurements from each binary, weighted by their uncertainties. The components of $\vec{x}_i$ will be determined by the difference between the two measured values; each component will have an uncertainty associated with it. For closely separated stars the angular separation can be determined precisely from the two stellar coordinates:
\begin{equation}
\theta \approx \sqrt{(\alpha_A - \alpha_B)^2 \cos \delta_A \cos \delta_B
			 + (\delta_A - \delta_B)^2}.
\end{equation}
The proper motion difference can be similarly calculated:
\begin{equation}
\Delta \mu \approx \sqrt{(\mu_{\alpha, A} - \mu_{\alpha, B})^2 
			\cos \delta_A \cos \delta_B 
			+ (\mu_{\delta, A} - \mu_{\delta, B})^2}.
\end{equation}
Uncertainties in $\Delta \mu$ are separately calculated from standard error analysis:
\begin{equation}
\sigma_{\Delta \mu}^2 = \left( \sigma_{\mu_{\alpha, A}}^2 + \sigma_{\mu_{\alpha, B}}^2 \right) \cos \delta_A \cos \delta_B + \sigma_{\mu_{\delta, A}}^2 + \sigma_{\mu_{\delta, B}}^2  
\end{equation}



Using these sets of parameters and the two classes, we can now use Bayes's theorem to construct the generalized probability that any pair of stars forms a true binary (class $C_2$):
\begin{equation}
P(C_2 \given \vec{x}_i, \vec{x}_j) = \frac{P(\vec{x}_i \given C_2, \vec{x}_j) P(C_2 \given \vec{x}_j)}{P(
\vec{x}_i \given \vec{x}_j)}. \label{eq:P_binary_1}
\end{equation}
The first term in the numerator is a likelihood while the second term in the numerator can be considered a prior probability of a binary with $\vec{x}_j$. The denominator in Equation \ref{eq:P_binary_1} is the evidence, which can be determined by summing the probability that the particular pair can be produced by both a binary ($C_2$) and a random alignment ($C_1$):
\begin{equation}
P(\vec{x}_i \given \vec{x}_j) = \sum_{k=1,2} P(\vec{x}_i \given C_k, \vec{x}_j) P(C_k \given \vec{x}_j).
\end{equation}


We can determine the prior probabilities, $P(C_1 \given \vec{x}_j)$ and $P(C_2 \given \vec{x}_j)$, based on our understanding of binaries and random alignments: $P(C_1 \given \vec{x}_j)$ should scale with $\rho^2(\vec{x}_j)$ while $P(C_2 \given \vec{x}_j)$ should scale with $\rho(\vec{x}_j)$:
\begin{eqnarray}
P(C_1 \given \vec{x}_j) &=& A_1\ \rho^2(\vec{x}_j) \\
P(C_2 \given \vec{x}_j) &=& A_2\ \rho(\vec{x}_j).
\end{eqnarray}
We can determine the coefficients of these proportionalities by integrating the relations over all four dimensions of $\vec{x}_j$ which will provide us with the total number of random alignments in the sample, $N(N-1)/2$, and the total number of binaries in the sample (including unresolved pairs), $f_{\rm bin} N$, where $f_{\rm bin}$ is the binary fraction:
\begin{eqnarray}
\frac{N(N-1)}{2} &=& \int \dd \vec{x}_j\ A_1\ \rho^2(\vec{x}_j) \\
N f_{\rm bin} &=& \int \dd \vec{x}_j\ A_2\ \rho(\vec{x}_j).
\end{eqnarray}
The integral to determine $A_1$ must be calculated numerically for a specific sample, while $A_2$ immediately becomes $f_{\rm bin}$:
\begin{eqnarray}
A_1 &\approx& \frac{1}{\int \dd \vec{x}_j\ P(\vec{x}_j)^2} \\
A_2 &=& f_{\rm bin},
\end{eqnarray}
where we have used the substitution that $\rho(\vec{x}_j) = N P(\vec{x}_j)$. We calculate the integral over $A_1$ using Monte Carlo random sampling over kernel density estimates of the distribution in position space and proper motion space. We describe these density estimators in detail in Section \ref{sec:random}.


We can now substitute the priors into Equation \ref{eq:P_binary_1} to obtain our posterior distribution:
\begin{equation}
P(C_2 \given \vec{x}_i, \vec{x}_j) = \frac{P(\vec{x}_i \given C_2, \vec{x}_j) f_{\rm bin} }{P(\vec{x}_i \given C_2, \vec{x}_j) f_{\rm bin}  + P(\vec{x}_i \given C_1, \vec{x}_j) A_1 \rho(\vec{x}_j)  }. \label{eq:P_binary_2}
\end{equation}




\subsection{Random Alignment Likelihood: $P(\vec{x}_i \given C_1, \vec{x}_j)$}
\label{sec:random}

The probability that a pair of stars with $\vec{x}_i$ and $\vec{x}_j$ could have been formed due to random alignments is $P(\vec{x}_i \given C_1, \vec{x}_j)$. We begin by marginalizing over the true individual parallaxes $\pi_1$ and $\pi_2$ and the true proper motion difference $\Delta \mu$ to account for observational uncertainties on these quantities:
\begin{equation}
P(\vec{x}_i \given C_1, \vec{x}_j) = \int \dd \pi_1\ \dd \pi_2\ \dd \Delta \mu\ P(\pi_1, \pi_2, \Delta \mu, \vec{x}_i \given C_1, \vec{x}_j). \label{eq:P_noise_marginalized}
\end{equation}
Now we can substitute for $\vec{x}_i$ and $\vec{x}_j$ and factor out $\pi'_1$, $\pi'_2$, and $\Delta \mu'$ since observed quantities are dependent only on their underlying values (and their associated uncertainties):
\begin{eqnarray}
P(\vec{x}_i \given C_1, \vec{x}_j) &=& \int  \dd \pi_1\ \dd \pi_2\ \dd \Delta \mu\ P(\pi'_1 \given \pi_1) P(\pi_1) \nonumber \\
	& &  \times P(\pi'_2 \given \pi_2) P(\pi_2) P(\Delta \mu' \given \Delta \mu) \nonumber \\
	& &  \times  P(\theta, \Delta \mu \given C_1, \alpha, \delta, \mu_{\alpha}, \mu_{\delta}). \label{eq:RA_substituted}
\end{eqnarray}
Our assumption that $\theta$ and $\Delta \mu$ are independent allow us to split the last term in the integrand. 
%\begin{eqnarray}
%P(\vec{x}_i \given C_1, \vec{x}_j) &=& \int  \dd \pi_1\ \dd \pi_2\ \dd \Delta \mu\ P(\pi'_1 \given \pi_1) P(\pi_1) \nonumber \\
%	& &  \times P(\pi'_2 \given \pi_2) P(\pi_2) P(\theta \given C_1, \alpha, \delta) \nonumber \\
%	& &  \times  P(\Delta \mu' \given \Delta \mu) P(\Delta \mu \given C_1, \mu_{\alpha}, \mu_{\delta}). \label{eq:RA_substituted}
%\end{eqnarray}
Furthermore, the integrals over $\pi_1$ and $\pi_2$ are independent and can be factored out. Equation \ref{eq:RA_substituted} therefore becomes:
\begin{eqnarray}
P(\vec{x}_i \given C_1, \vec{x}_j) &=& \int P(\theta \given C_1, \alpha, \delta) P(\Delta \mu' \given \Delta \mu) \nonumber \\
	& & \times\ P(\Delta \mu \given C_1, \mu_{\alpha}, \mu_{\delta})\ \dd \Delta \mu \nonumber \\
	& &  \times \int  P(\pi'_1 \given \pi_1) P(\pi_1)\ \dd \pi_1 \nonumber \\
	& &  \times \int  P(\pi'_2 \given \pi_2) P(\pi_2)\ \dd \pi_2. \label{eq:RA_factored}
\end{eqnarray}

To determine the $\theta$ term in Equation \ref{eq:RA_factored}, we recognize that as $\theta$ increases, the probability of random alignments increases linearly. This scaling is because, for a uniformly dense population on the sky, consecutively larger annuli surrounding a star have areas that increase linearly, and the probability of random alignments should scale with the area (at least locally, so the density of stars per phase space is constant). The scaling coefficient depends on the local stellar density surrounding a point. We therefore determine this scaling empirically. 

To do this, we calculate the kernel density estimate (KDE), using a Gaussian kernel, for the distribution of stars in $(\alpha \cos \delta, \delta)$-space in the Gaia-Tycho-2 catalog. The KDE acts as a smoothing function over the distribution in Figure \ref{fig:tycho-2_pos_mu} and produces a normalized probability density function over position space. 

%\begin{figure}
%\begin{center}
%\includegraphics[width=0.95\columnwidth]{../figures/tycho-2_local_density.pdf}
%\caption{ For one particular star, the solid line in the top panel shows the normalized histogram of the distances to other stars in the Tycho-2 catalog. For a constant density (locally, at least) the distribution should rise linearly with angular separation. The dashed line shows the expectation from our KDE estimate using Equation \ref{eq:P_theta}. For the same star, the solid line in the bottom panel shows the equivalent normalized histogram of proper motion differences. For proper motion distributions beyond $\approx$10 mas yr$^{-1}$, the distribution substantially rises, deviating from linear. The dashed line again indicates our approximation using the KDE density estimate using Equation \ref{eq:P_mu}. Since typical proper motion uncertainties are $\lesssim$ 5 mas yr$^{-1}$, our linear approximation is reasonable. Distributions for other stars are similar to the ones shown here.}
%\label{fig:tycho-2_density}
%\end{center}
%\end{figure}

The top panel of Figure \ref{fig:tycho-2_density} demonstrates the typical accuracy of our empirically-derived estimate for the local space density. This figure shows the distribution for one particular star representative of typical accuracies for our estimation methods. The solid line indicates the normalized histogram of stars at a distance $\theta$, while the dashed line approximates this distribution using our KDE estimate. The dashed line is determined independently for each primary object by calling the KDE function to determine the stellar density, as a function of $\alpha$ and $\delta$:
\begin{equation}
P(\theta \given C_1, \alpha, \delta) = 2 \pi \rho(\alpha, \delta)\ \theta. \label{eq:P_theta}
\end{equation}


The $\Delta \mu$ term in Equation \ref{eq:RA_factored} provides the corresponding effect for proper motion. This term can also be calculated empirically using a KDE, again with a Gaussian kernel. The bottom panel of Figure \ref{fig:tycho-2_density} demonstrates the accuracy of our estimation method for the number of stars with similar proper motions to the same star as in the top panel. The solid line shows the actual normalized histogram of proper motion difference $\Delta \mu$, while the dashed line shows the KDE estimate: 
\begin{equation}
P(\Delta \mu \given C_1, \alpha, \delta) = 2 \pi \rho_{\mu}(\mu_{\alpha}, \mu_{\delta})\ \Delta \mu, \label{eq:P_mu}
\end{equation}
where $\rho_{\mu}(\mu_{\alpha}, \mu_{\delta})$ is the proper motion-dependent density determined by the KDE.

Due to smaller scale variations in the proper motion, it is evident from Figure \ref{fig:tycho-2_density} that the KDE estimation method deviates from the true distribution for proper motion differences $>10$ mas yr$^{-1}$. Since typical binaries have $\Delta \mu \lesssim$ 5 mas yr$^{-1}$, the KDE method is accurate enough for our purposes here. 




Probability terms involving the observed quantities in Equation \ref{eq:RA_factored} can be substituted as normal distributions, based on the Gaussian uncertainties in these terms. Non-Gaussian uncertainties can be taken into account here as well:
\begin{eqnarray}
P(\Delta \mu' \given \Delta \mu) &=& \mathcal{N}(\Delta \mu \given \Delta \mu', \sigma^2_{\Delta \mu}) \label{eq:delta_mu_uncertainty} \\
P(\pi'_1 \given \pi_1) &=& \mathcal{N}(\pi_1 \given \pi'_1, \sigma^2_{\pi_1}) \label{eq:pi_1_uncertainty} \\
P(\pi'_2 \given \pi_2) &=& \mathcal{N}(\pi_2 \given \pi'_2, \sigma^2_{\pi_2}). \label{eq:pi_2_uncertainty} \\
\end{eqnarray}


{\bf Paragraph and plot on parallaxes. Do we need to consider the Lutz-Kelker bias in $P(\pi)$ here?}


The random alignment likelihood, $P(\vec{x}_i \given C_1, \vec{x}_j)$ can now be determined by substituting  Equations \ref{eq:P_theta}, \ref{eq:P_mu}, \ref{eq:delta_mu_uncertainty}, \ref{eq:pi_1_uncertainty}, and \ref{eq:pi_2_uncertainty} into the marginalized and factored Equation \ref{eq:RA_factored}. We calculate these integrals using Monte Carlo sampling; our convergence tests indicate that 10$^3$ samples provide a sufficient accuracy for our purposes here.




Equation \ref{eq:RA_factored}, with our empirical estimate based on the local stellar density for $P(\theta \given C_1, \alpha, \delta)$ and the KDE estimate for $P(\Delta \mu \given C_1, \alpha, \delta, \mu_{\alpha}, \mu_{\delta})$ then defined the likelihood for any particular pair of stars to be produced by random alignment.



\begin{comment}
With a detailed knowledge of the three-dimensional stellar density, proper motion distribution as a function of spatial position, and completeness function of the Gaia-TYCHO-2 sample, one could determine $P(\theta, \Delta \mu, \Delta D' \given C_1, \alpha, \delta, \mu_{\alpha}, \mu_{\delta}, D )$ for any arbitrary pair of stars. However, this is a function with complex dependencies that we are not prepared to address here. Instead, we can empirically approximate this probability for every pair of stars in our data set. First, we express $\mu$ as $V_{\rm pec}$ (to remove the correlation between proper motion and distance), and then separate the terms in the probability:
\begin{eqnarray}
P(\theta, \Delta \mu, \Delta D' \given C_1, \alpha, \delta, \mu_{\alpha}, \mu_{\delta}, D ) &=& P(\theta, \Delta V_{\rm pec}, \Delta D' \given C_1, \alpha, \delta, \mu_{\alpha}, \mu_{\delta}, D ) \left| \frac{\dd \Delta V_{\rm pec}}{\dd \Delta \mu} \right| \label{eq:P_obs_noise} \\
&=& P(\theta \given \alpha, \delta) 
  P(\Delta V_{\rm pec} \given \alpha, \delta, \mu_{\alpha}, \mu_{\delta}) \nonumber \\
  & & \qquad \times  P(\Delta D' \given \alpha, \delta, D) \left| \frac{\dd \Delta V_{\rm pec}}{\dd \Delta \mu} \right|. \label{eq:P_obs_noise_split} 
\end{eqnarray}

These terms can be separated so long as no correlation exists between them. There is likely to be some dependence between these terms; for instance at larger distances, farther from the Galactic plane, stars are more likely to be halo members with larger peculiar velocities. A more complex model could account for this, however here we consider such variations to have only a minor effect on the probability in Equation \ref{eq:P_obs_noise}. 

We can determine the three probabilities in Equation \ref{eq:P_obs_noise_split} empirically using the following procedure. First, we select an arbitrarily large radius around each star to search for matching pairs. We then numerically generate three normalized histograms, one for each observable: $\theta$, $\Delta V_{\rm pec}$, and $\Delta D'$. Interpolating along the histograms, for each observable, for each pair, provides the three probabilities in Equation \ref{eq:P_obs_noise_split}. What remains is to determine a large enough search radius, such that there are enough that even widely separated, genuine stellar pairs can be resolved and that there are enough stellar pairs such that the three histograms do not suffer from low number statistics. For separations too large, our assumption of independence in observables breaks down, and the split we made in Equation \ref{eq:P_obs_noise} is no longer justified. This search radius needs to be calibrated, but for now, we suggest that 1\degree should be sufficient.

Combining Equation \ref{eq:P_obs_noise_split} and \ref{eq:P_noise_marginalized} gives us the following:
\begin{eqnarray}
P(\vec{x}_i \given C_1, \vec{x}_j) &=& P(\theta \given \alpha, \delta)\ 
   P(\Delta V_{\rm pec} \given \alpha, \delta, \mu_{\alpha} \mu_{\delta})\ 
   \left| \frac{\dd \Delta V_{\rm pec}}{\dd \Delta \mu} \right| \nonumber \\
   & & \qquad \times  \int \dd \Delta D'\ P(\Delta D \given \Delta D')\ P(\Delta D' \given \alpha, \delta, D),
\end{eqnarray}
where we were able to move the conditional probabilities for $\theta$ and $\Delta V_{\rm pec}$ outside the integral over $\Delta D'$ because of their independence on distance.
\end{comment}




\subsection{Binary Likelihood: $P(\vec{x}_i \given C_2, \vec{x}_j)$}

We now determine the probability that a true binary could produce the observations, $P(\vec{x}_i \given C_2, \vec{x}_j)$. We begin by first accounting for quantities with observational uncertainties. We marginalize over the true parallax, the true proper motion difference, the tangential velocity difference, $\Delta V$, and projected physical separation, $s$. Both components of a true binary will have the same distance (the widest binaries have separations $<$1 pc, less than Gaia's nominal parallax distance uncertainty), so we need only marginalize over one parallax:
\begin{eqnarray}
P(\vec{x}_i \given C_2, \vec{x}_j) &=& \int \dd \Delta \mu\ \dd \pi\ \dd \Delta V\ \dd s\ \nonumber \\
& & \qquad \times P( \Delta \mu, \pi, \Delta V, s,\vec{x}_i \given C_2, \vec{x}_j ). \label{eq:P_binary_marginalized_1}
\end{eqnarray}
Observed quantities depend only on the underlying values. We assume that the population of binaries is independent of both position in the Galaxy and its peculiar velocity, so $s$ and $\Delta V$ depend only on $C_2$. Upon substitution for $\vec{x}_i$ and $\vec{x}_j$, Equation \ref{eq:P_binary_marginalized_1} becomes:  
\begin{eqnarray}
P(\vec{x}_i \given C_2, \vec{x}_j) &=& \int \dd \Delta \mu\ \dd \pi\ \dd \Delta V\ \dd s\ P(\Delta \mu' \given \Delta \mu) \nonumber \\
& & \qquad \times P(\pi'_1 \given \pi)\ P(\pi'_2 \given \pi)\ P(s, \Delta V \given C_2) \nonumber \\
& & \qquad \times P(\Delta \mu \given \Delta V, \pi)\ P(\theta \given s, \pi)\ P(\pi), \label{eq:P_binary_marginalized_2}
\end{eqnarray}
where we have 

The fifth and sixth terms in the integrand of Equation \ref{eq:P_binary_marginalized_2} are delta functions that account for the proper motion difference and projected separation:
\begin{eqnarray}
P(\Delta \mu \given \Delta V, \pi) &=& \delta \left[ F(\Delta V) \right] \\ 
P(\theta \given s, \pi) &=& \delta \left[ G(s) \right], 
\end{eqnarray}
where $F(\Delta V) = \Delta \mu - \Delta V \pi$ and $G(s) = \theta - s \pi$. 

These delta functions reduce the integrals over $\Delta V$ and $s$ in Equation \ref{eq:P_binary_marginalized_2}:
\begin{eqnarray}
P(\vec{x}_i \given C_2, \vec{x}_j) &=& \int \dd \Delta \mu\ \dd \pi\ P(\Delta \mu' \given \Delta \mu)\ P(\pi'_1 \given \pi) \nonumber \\
& & \qquad \times P(\pi'_2 \given \pi)\ P(s, \Delta V \given C_2)\ P(\pi) \nonumber \\
& & \qquad \times  \left| \frac{\dd F(\Delta V^*)}{\dd \Delta V} \right|^{-1} \left| \frac{\dd G(s^*)}{\dd s} \right|^{-1}, \label{eq:P_binary_marginalized}
\end{eqnarray}
where $\Delta V^*$ and $s^*$ are the roots to $F(\Delta V)$ and $G(s)$, respectively. These last two terms in the integrand both reduce to $\pi^{-1}$.




The first three terms of the integrand are the observed distributions which we assume are Gaussian:
\begin{eqnarray}
P(\pi'_1 \given \pi) &=& \mathcal{N}(\pi'_1 \given \pi, \sigma^2_{\pi}) \\
P(\pi'_2 \given \pi) &=& \mathcal{N}(\pi'_2 \given \pi, \sigma^2_{\pi}) \\
P(\Delta \mu' \given \Delta \mu) &=& \mathcal{N}( \Delta \mu' \given \Delta \mu, \sigma^2_{\Delta \mu} ).
\end{eqnarray}



The fourth term in Equation \ref{eq:P_binary_marginalized}, $P(s, \Delta V_{\rm pec} \given C_2)$, expresses the likelihood that a random binary would produce the observed projected separation and velocity difference. In general, this function depends on assumptions made about populations of binary stars; we assume a binary is completely determined by four parameters: the two stellar masses, $M_1$ and $M_2$, the orbital separation, $a$, and the eccentricity, $e$. We do not include any binary evolution interactions between the two stars. 


We determine the density of true binaries in $s-\Delta V$ space by randomly generating 10$^6$ binaries from the above distributions, with random orientation angles. Figure \ref{fig:P_binary} demonstrates the distribution of our randomly generated binaries in this space. The bottom and left axes indicate $s$ and $\Delta V$, respectively, while the top and right axes indicate $\theta$ and $\Delta \mu$, respectively, if the entire population were at a distance of 100 pc. Gaia, with its ability to discern between double stars with subarcsecond separations and its proper motion precision of $\sim$ 1 mas yr$^{-1}$ for the joint Gaia-Tycho-2 catalog, can indeed detect differences in the orbital motion of the components of these widely separated binaries.


From the population of binaries shown in Figure \ref{fig:P_binary}, we use a KDE with a Gaussian kernel to create a normalized probability density function in this space. Evaluating the KDE at a particular $s$ and $\Delta V$ provides $P(s, \Delta V_{\rm pec} \given C_2)$.



The final term in Equation \ref{eq:P_binary_marginalized} is the Lenz-Kelker bias on the parallax measure. Equation \ref{eq:P_binary_marginalized} can then become:
\begin{eqnarray}
P(\vec{x}_i \given C_2, \vec{x}_j) &=& \int \dd \Delta \mu\ \dd \pi\ 
\mathcal{N}( \Delta \mu' \given \Delta \mu, \sigma^2_{\Delta \mu} ) \nonumber \\
& & \qquad \times\ \mathcal{N}(\pi'_1 \given \pi, \sigma^2_{\pi})\ 
	\mathcal{N}(\pi'_2 \given \pi, \sigma^2_{\pi}) \nonumber \\
& & \qquad \times\ P(s, \Delta V \given C_2)\ P(\pi)\ \pi^{-2}.
\label{eq:P_binary}
\end{eqnarray}
We solve these integrals simultaneously using Monte Carlo random sampling. Convergence tests indicate that 10$^3$ random samples provides sufficient accuracy for our purposes here.









\section{Application to the Revised NLTT Catalog} \label{sec:rNLTT}


To test our method, we apply this method to the revised NLTT (rNLTT) catalog \citep{gould03, salim03}, containing 36,085 entries with a typical proper motion uncertainty of several mas yr$^{-1}$. A comparison of the pairs we identify to the 999 wide binaries previously identified by \citet[][hereafter CG04]{chaname04} in the rNLTT catalog provides a test of our method. The comparison is imperfect, since the rNLTT sample lacks the parallax measurements within the Gaia-Tycho-2 catalog. To apply our method to this sample, we assume all stars in the sample are at a distance of 100 pc, yet as we will see below, despite such a clearly incorrect assumption, our method remains effective.




There are some important differences between the two search methods that deserve mention. First, CG04 identified binaries by first making simple proper motion cuts, using two separate $\Delta \mu$ limits for disk and halo binaries. Proper motion uncertainties are ignored, whereas the method defined here both includes uncertainties in proper motion as well as a quantification of binary probability as a function of $\theta$, $\Delta \mu$, and position in stellar position$-$proper motion phase space. However, to remove potential contaminants, CG04 include a constraint using an adapted version of the reduced proper motion to discern between stellar disk and halo populations. 
%\begin{equation}
%\eta = V + 5 \log \mu - 3.1(V-J) - 1.47 | \sin b | - 2.73. \label{eq:eta_rpm}
%\end{equation}
Clearly, true pairs cannot contain stars from different populations, a powerful constraint that our method ignores (Gaia-Tycho-2 parallaxes will make this constraint obsolete). Given the differences in these two methods, pairs disjoint between the two samples are to be expected.



Since we set the distance to every star in the catalog to 100 pc, our method greatly simplifies. $\vec{x}_i$ is redefined as $\vec{x}_i = \{ \theta, \Delta \mu' \}$. Equation \ref{eq:RA_substituted} for $P(\vec{x}_i \given C_1, \vec{x}_j)$ becomes:
\begin{eqnarray}
P(\vec{x}_i \given C_1, \vec{x}_j) &=& \int \dd \Delta \mu\ P(\Delta \mu' \given \Delta \mu)\nonumber \\
& & \qquad \times P(\theta \given C_1, \alpha, \delta) P(\Delta \mu \given C_1, \alpha, \delta, \mu_{\alpha}, \mu_{\delta}) \nonumber \\
\label{eq:P_random_NLTT}
\end{eqnarray}


Equation \ref{eq:P_binary} for $P(\vec{x}_i \given C_2, \vec{x}_j)$ becomes:
\begin{equation}
P(\vec{x}_i \given C_2, \vec{x}_j) = \int \dd \Delta \mu\ 
\mathcal{N}( \Delta \mu' \given \Delta \mu, \sigma^2_{\Delta \mu} )\
P(s, \Delta V \given C_2),
\label{eq:P_binary_rNLTT}
\end{equation}
where $s = \theta D$ and $V = \Delta \mu D$. 


Through their search algorithm, they find 1147 candidate pairs in the rNLTT catalog. In our search through the rNLTT catalog using our algorithm here, we identify 1189 pairs in the rNLTT catalog with a posterior probability $>50$\%. Although the yield of wide binaries is remarkably similar despite very different search algorithms, the overlap in the two samples (number of redetections) is only 620 pairs. 



%\begin{figure}
%\begin{center}
%\includegraphics[width=0.95\columnwidth]{../figures/rNLTT_CG04_pos_mu.pdf}
%\caption{ The top panel compares the $\theta$ and $\Delta \mu$ distribution expected from a set of wide binaries in blue contours (at 100 pc) with wide binaries identified by CG04. Pairs that we find have $P(C_2 \given \vec{x}_i, \vec{x}_j) > 50\%$ (filled, black circles) typically have smaller $\theta$ and $\Delta \mu$, while pairs which our method rejects (open, red circles) are found at larger separations and proper motion differences. The bottom panel indicates those pairs identified by our method that were undetected by CG04. Uncertainties on $\Delta \mu$ are typically $\approx$10 mas yr$^{-1}$. }
%\label{fig:rNLTT_theta_mu}
%\end{center}
%\end{figure}



%\begin{figure*}
%\begin{center}
%\includegraphics[width=0.95\textwidth]{../figures/rNLTT_probs.pdf}
%\caption{ The left-most panel shows the distribution of posterior probabilities for the pairs we detect (solid line) as well as the pairs in the CG04 catalog that are identified and accepted (dotted line) and identified and rejected (dashed line). Our method typically accepts or rejects pairs of stars with very high likelihood; there are relatively few pairs with posterior probabilities not close to either zero or unity. The middle and right panels show distributions for the same samples of binary likelihoods and random alignment likelihoods, respectively. Although there is a substantial overlap of 620 re-detected pairs, our model identifies an additional sample of 564 binaries missed by CG04 with lower random alignment likelihoods and higher binary likelihoods on average. }
%\label{fig:NLTT_probs}
%\end{center}
%\end{figure*}



The differences in the two samples are illustrative. Of the 1147 CG04 pairs, 80 pairs have one or both stars missing from the rNLTT catalog. A further 339 pairs have identical positions, proper motions, or one or both stars without proper motion uncertainties for one of the components. This typically occurs because Luyten identified the pair as a true binary and used characteristics of both stars in determining the system's astrometry \citep{chaname04}. By removing these stars when selecting a clean catalog to search through, we are insensitive to these binaries, which are likely true pairs. We are left with 728 pairs of stars in the CG04 catalog which our algorithm should detect. The $\theta$ and $\Delta \mu$ distribution of these binaries with posterior probabilities above 50\% (filled, black circles) and with probabilities below 50\% (open, red circles) are shown in the top panel of Figure \ref{fig:rNLTT_theta_mu}. 


253 pairs in CG04 are rejected as they have either separations or proper motions too large for our algorithm to identify with confidence that these are true binaries. These pairs have angular separations extending out to several 100 mas yr$^{-1}$, although in Figure \ref{fig:rNLTT_theta_mu} we only focus on those pairs within 150 mas yr$^{-1}$. In some of these cases, the posterior probability is near, but just below, 50\% in which case our assumed distance of 100 pc (or 50\% binary fraction) could responsible for our non-detection. Astrometric parallax would be helpful for identifying these marginal cases in the future. As such, the results from our method here should be taken as a demonstration of how our method compares to previous searches for wide binaries rather than a conclusive statement as to the nature of these pairs.



In addition to the 620 redetections, our method identifies five detected pairs were components of pairs in the CG04 catalog, but identified with different companions by our method; each of these five were found to be part of a triple system, in which our method matched a different pair of stars within the triple. Actually there were four triples, but one was matched twice producing five pairs. {\bf There are undoubtedly other triples in this sample. Write a code block to search specifically for higher order systems.}


Finally, we identify an additional 564 stellar pairs with posterior probabilities above 50\% that were not identified by CG04. Only 132 of these have independently measured positions and proper motions. These binaries are all detected with high significance, and we therefore suspect they are genuine pairs. The bottom panel of Figure \ref{fig:rNLTT_theta_mu} shows the distribution of these pairs in $\theta$ and $\Delta \mu$. The majority of these pairs, unidentified by CG04, lack $B$ and $J$ photometry and were therefore removed by CG04 in their attempt to identify a set of wide binaries from a heterogeneous data set.

%44 of these are in the CG04 catalog, but ultimately ruled out as false positives for various reasons (22 because of ``uncertain classification", 15 because of ``inconsistent RPM positions", and seven because of ``large proper motion differences"). 




Separately, we run our detection algorithm on the sample by CG04 and compare these probabilities to the pairs we identify in the rNLTT catalog. The first panel of Figure \ref{fig:NLTT_probs} shows the posterior probabilities for the set of binaries we detect (solid lines), detected and accepted by CG04 (dotted lines), and detected but ultimately rejected by CG04 (dashed lines). The distribution of posterior probabilities for the CG04 pairs indicates that binaries are typically either detected or rejected with a high likelihood by our method; there are relatively few marginal cases with posterior probabilities near 50\%.


The middle and right panels of Figure \ref{fig:NLTT_probs} show distributions for the same samples of binary likelihoods and random alignment likelihoods, respectively. Although there is a substantial overlap of 620 re-detected pairs, our model identifies an additional sample of 564 binaries missed by CG04 with lower random alignment likelihoods and higher binary likelihoods on average. 





\begin{comment}

Gray points in each panel of Figure \ref{fig:NLTT_probabilities} show the distribution of $P(\vec{x}_i \given C_1, \vec{x}_j)$ and $P(\vec{x}_i \given C_2, \vec{x}_j)$ for the pairs we identify. Comparison with the dashed lines corresponding to posterior probabilities of 50\% and 90\% indicate that the majority of these binaries are detected with very high significance. 

Of the sample identified by CG04, we redetect 707 pairs ($\approx 70$\%) with a posterior probability greater than 50\%. Colored points in the four panels of Figure \ref{fig:NLTT_probabilities} compare the pairs in CG04 with all 946 pairs that we identify, indicated as gray points. The top left panel shows the disk binaries, the top right panel shows the binaries containing a WD, the bottom left panel shows the halo binaries, and the bottom right binary shows those binaries detected but ultimately rejected by CG04 for reasons provided in the legend. Differences between the two samples are substantial.  

%\begin{figure*}
%\begin{center}
%\includegraphics[width=0.95\textwidth]{../figures/rNLTT_P_binary_P_random.pdf}
%\caption{ {\bf Unclear if this is the best figure to use to compare our pairs with those in Chaname \& Gould (2004). It is clear there will be differences, so we should focus on what about those differences we want to highlight.} Likelihoods of stellar pairs being produced by a true binary and a random alignment. Gray points indicate the 949 pairs with posterior probabilities of being a true binary greater than 50\%. Colored points in the four panels indicate disk binaries (top left), WD binaries (top right), halo binaries (bottom left), and binaries identified but rejected by CG04 (bottom right). }
%\label{fig:NLTT_probabilities}
%\end{center}
%\end{figure*}

\end{comment}





%\begin{figure}[h!]
%\begin{center}
%\includegraphics[width=0.95\columnwidth]{../figures/}
%\caption{ }
%\label{fig:test_corner}
%\end{center}
%\end{figure}


\begin{comment}

\section{Application to the Tycho-2 Catalog}

We apply our method to the Tycho-2 catalog, prior to its astrometric updates from {\it Gaia}. Figure \ref{fig:tycho-2_color_mag} demonstrates a distribution of a subset of the stars in the Tycho-2 catalog. We also indicate colors corresponding to different Main Sequence stars (black, dashed) and lines of constant distance (red, dashed) for Main Sequence stars. The bulk of the Main Sequence stars in the Tycho-2 catalog are at distances exceeding 100 pc. As with the rNLTT catalog, we lack astrometric parallax measurements, therefore we set all stars to a distance of 100 pc. Since the probability of any pair of stars being a true binary increases with closer distances, and Figure \ref{fig:tycho-2_color_mag} indicates that most stars are at a distance greater than 100 pc, our approximation overestimates the binary probability of pairs.


From the $2.5\times10^6$ stars in the catalog, we select those that have measured proper motions, leaving us with $\sim2\times10^6$ stars to search from. This set is nearly two orders of magnitude larger than the rNLTT catalog, and designing a computationally efficient algorithm is crucial to making the identification of wide binaries a tractable problem. Our method is as follows. For every star, we select each of the nearby stars within 1\degree. For typical stellar densities in the Tycho-2 sample, this corresponds to $\approx50$ stars. For each of these potential pairs, we apply a second criterion: we remove pairs with a binary likelihood, $P(\vec{x}_i \given C_2, \vec{x}_j)$, equal to zero $\Delta \mu$ at the 3-$\sigma$ limit and $\theta$. For pairs with non-zero binary likelihoods resulting, we then calculate the binary likelihood (using 10$^4$ Monte Carlo random samples for the integral of $\Delta \mu$) and the random alignment likelihood. We select pairs with posterior probabilities of being a binary at greater than 50\%.
 
 
 
%\begin{figure}
%\begin{center}
%\includegraphics[width=0.95\columnwidth]{../figures/tycho2_colors.pdf}
%\caption{ The color-magnitude distribution of a random subset of the Tycho-2 catalog. Vertical dashed lines (black) indicate colors roughly corresponding to the stellar types labeled at the bottom of the figure for Main Sequence stars. Since reddening is ignored, these are only approximate. Slanted dashed lines (red) indicate corresponding distances for Main Sequence stars. The bulk of the Main Sequence stars are at distances exceeding 100 pc.}
%\label{fig:tycho-2_color_mag}
%\end{center}
%\end{figure}



Figure \ref{fig:tycho-2_theta_mu} demonstrates the first results from our search for binaries in the Tycho-2 catalog. We find some 8500 pairs with a posterior probability of being a true binary above 50\% from the first 20\% of the catalog. This implies there are some 40,000 wide binaries in the true catalog. This remains to be seen. Many or even most of these binaries may be random alignments, which astrometric parallax can help discern. 

Nevertheless, from Figure \ref{fig:tycho-2_theta_mu} it is clear the systematics involved with finding binaries in the Tycho-2 Catalog are substantially different from the rNLTT catalog. Principally, the proper motion separation is substantially more dense in the Tycho-2 catalog. At the same time, the Tycho-2 catalog has a typical uncertainty on $\Delta \mu$ of $\approx4$ mas yr$^{-1}$, substantially lower than the rNLTT catalog. Nevertheless, we expect the rate of false positives to be much higher. These random alignments could be the cause of the increased density of identified matches at angular separations $>$50\asec, which is unexpected to be produced by a population of genuine binaries. This leads us to one of our primary conclusions: {\it even with proper motion precisions of order a few mas yr$^{-1}$, samples of common proper motion binaries identified in low proper motion catalogs will contain a large number of false positives. } {\bf Previous statement may need to be made later in paper when we can actually separate out true from false pairs using Gaia astrometry.} An additional dimension of phase space is required to separate true binaries from unassociated stellar pairs.



\begin{figure}
\begin{center}
\includegraphics[width=0.95\columnwidth]{../figures/tycho-2_theta_mu.pdf}
\caption{ The $\theta-\Delta \mu$ distribution of the first set of pairs in the Tycho-2 catalog. Since we have assumed all stars are at a distance of 100 pc, there is a sharp cut-off in the distribution of angular separations of identified binaries. We compare these stars to the distribution of binary probabilities in blue contours. Uncertainties on $\Delta \mu$ are typically $\approx$4 mas yr$^{-1}$. }
\label{fig:tycho-2_theta_mu}
\end{center}
\end{figure}



Figure 





\begin{figure}
\begin{center}
\includegraphics[width=0.95\columnwidth]{../figures/tycho-2_tmp.pdf}
\caption{ The distribution of $P(vec{x}_i \given C_2, \vec{x}_j)$ and $P(\vec{x}_i \given C_1, \vec{x}_j)$ for the $\approx$8500 matches in the first 20\% of the Tycho-2 catalog. Lines indicating posterior probabilities of 50\% and 90\% are indicated by the red, dashed lines. The large number of samples between these two lines indicates that many matches are only marginally detected, and therefore sensitive to our model assumptions. }
\label{fig:tycho-2_tmp}
\end{center}
\end{figure}



\end{comment}



\section{Application to the Gaia-Tycho-2 Catalog}

{\bf This will be the meat of the paper.}

\subsection{Identifying Wide Binaries}
{\bf Apply our method to the Gaia-Tycho-2 catalog. Description of application to catalog here. Overall yields.}

{\bf Figure: scatter plot of distance vs. primary's magnitude; many more binaries are expected in future data releases. This is only the tip of the iceberg.}

{\bf Table here. Only show first 10 entries.}

\subsection{Consistency Checks}

Hopefully, we find that all our binaries are genuine. 

{\bf Cross-matching with RAVE.}

{\bf RPM diagram goes here, too.}


\subsection{Angular and Physical Separation of Binaries}
{\bf Raw counts/distributions here will be interesting enough. Accounting for luminosity or method-biases will be very annoying and difficult, so leave this for a future work? Separation distributions as a function of stellar type would be interesting.}


{\bf Corner plot here: $\theta$ vs. $\mu$. Histograms on top and right. }

{\bf Corner plot here: $s$ vs. $V_{\rm trans}$. Histograms on top and right.}

Make it absolutely, crystal clear that these distributions are biased in unclear ways.


\subsection{Determining Stellar Type/Classification}

{\bf Binaries can be identified by their position on a color-magnitude diagram.}

{\bf Binaries with similar types (twin binaries). Binaries with subdwarfs or giants. WD wide binaries. Provide these distributions of stellar combinations, maybe as a table?}

{\bf Triangle table showing number of binaries of each stellar type.}


\subsection{Specific Nearby Wide Binaries}
{\bf Redetect binaries in \citet{shaya11}? Find other nearby binaries that we should list specifically. This may be better presented in an appendix.}




\subsection{Identified Triple Systems} 
{\bf We already know there are some. Let's focus on just astrometric triples - there will be lots of spectroscopic binaries as components of our wide binaries, but this will require substantial cross matching, I think.}

{\bf Stick with just our results; astrometric triples. Are they clearly hierarchical?}

{\bf Figure: scatter plot showing smallest and middle angular separations between components.}




\section{Discussion}


\subsection{Wide binary populations}
{\bf Identify kinematically different populations: Halo vs. disk?}


\subsection{The Widest binaries}
{\bf We are, in principle, sensitive to *extremely* wide binaries using this method. If there's anything near a parsec, or even a tenth of a parsec, we should discuss. These tell us about distributions of stars, giant molecular clouds, the Galactic tide, and dark matter distribution of the Milky Way. }


\subsection{Application to Future Gaia Data Releases}
{\bf 10$^9$ stars with full 5-D astrometry. What about subset with 6-D astrometry? Discuss computation expense challenges. Speculate on methods of increasing algorithm efficiency. Extrapolate this sample to estimate numbers of binaries expected from full sample. } 





\section{Conclusions}

{\bf Overview of our approach and results.}

{\bf New data will be coming soon (2017).}

{\bf Mention some ideas about science that can be done with large wide binary data sets. In particular, can we get a luminosity-unbiased sample, or at least account for observational biases (and biases introduced by detection methods).}


\section*{Acknowledgements}
Acknowledgements:
It is a pleasure to thank...
Funding...
Code...


\bibliographystyle{mnras}
\bibliography{references}

\label{lastpage}

\end{document}
